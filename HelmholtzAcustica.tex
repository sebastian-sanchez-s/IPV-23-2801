\documentclass[11pt]{article}

% Layout
\usepackage[margin=3cm]{geometry}

% Tools
\usepackage[colorlinks=true]{hyperref}
\usepackage{enumerate}
\usepackage{enumitem}
\usepackage{mathtools}
\usepackage{multirow}
\usepackage[spanish]{babel}
\usepackage{tikz}
\usetikzlibrary{babel}
\usetikzlibrary{calc}
\usetikzlibrary{patterns}
\usetikzlibrary{arrows.meta}

% Custom design
\usepackage{titlesec}
\usepackage{float}
\usepackage{caption}

% Math, Fonts and Symbols
\usepackage{mathptmx}
\usepackage{amsmath, amsthm, amssymb, bm}
\usepackage[T1]{fontenc}

%%%%%%%%%%%%%%%%%%%%%%%%%%%%%%%%%%%%%%%%%%%%%%%
% Title fomat
%%%%%%%%%%%%%%%%%%%%%%%%%%%%%%%%%%%%%%%%%%%%%%%

\renewcommand{\thesection}{\arabic{section}}
\numberwithin{equation}{section}

% Section
\titleformat{\section}[block]
{\large\bfseries\scshape}% format
{\thesection.}% label
{0.5em}% sep
{}% before code
[\titlerule] % after code

%%%%%%%%%%%%%%%%%%%%%%%%%%%%%%%%%%%%%%%%%%%%%%%
% Custom commands and Operators
%%%%%%%%%%%%%%%%%%%%%%%%%%%%%%%%%%%%%%%%%%%%%%%

% Derivative
\let\d\relax
\newcommand{\d}[1]{\,\frac{\mathrm{d}}{\mathrm{d}#1}}
\newcommand{\dd}[1]{\,\frac{\mathrm{d}^2}{\mathrm{d}#1}}

% Sets
\def\R{\mathbb{R}}
\def\C{\mathbb{C}}
\def\H{\mathcal{H}}
\def\Cc{\mathcal{C}}

% Bold vector
\def\x{\textbf{x}}
\def\y{\textbf{y}}
\def\z{\textbf{z}}

% Normal vector
\def\n{\hat{\textbf{n}}}

% Direction vector
\def\dir{\textbf{d}}

% Imaginary unit
\let\i\relax
\def\i{\mathrm{i}}

% Operators
\def\Slpot{\mathcal{V}}
\def\Dlpot{\mathcal{K}}

\def\Slop{\mathrm{V}}
\def\Dlop{\mathrm{K}}
\def\Adlop{\mathrm{T}}
\def\Hop{\mathrm{D}}

\def\Idop{\textrm{I}}

\def\traceN{\gamma_{N}}
\def\traceD{\gamma_{D}}

% subcripts
\def\tot{\textrm{tot}}
\def\exterior{\textrm{ext}}
\def\interior{\textrm{int}}
\def\inc{\textrm{inc}}
\def\sca{\textrm{sca}}

% Letters
\def\Rho{\mathrm{P}}

% Resize abs and norm
\DeclarePairedDelimiter{\abs}{\lvert}{\rvert}
\DeclarePairedDelimiter{\norm}{\|}{\|}
\makeatletter
\let\oldabs\abs
\def\abs{\@ifstar{\oldabs}{\oldabs*}}
\let\oldnorm\norm
\def\norm{\@ifstar{\oldnorm}{\oldnorm*}}
\makeatother

\title%
{\bfseries\scshape%
Dispersión de la onda acústica plana en la esfera homogénea\\
\normalsize Soluciones analíticas y formulación BEM
}
\author{\bfseries\normalsize Sebastián Sánchez}
\date{}

\begin{document}

\maketitle

\tableofcontents

\section{Introducción}

Los fenómemos oscilatorios son comunes al pertubar un sistema estable;
Tal es el caso, por ejemplo, de una cuerda, un péndulo e incluso la presión 
del aire generada por el sonido. Todos estos fénonemos pueden ser descritos por
la ecuación de ondas
\begin{displaymath}
	\partial^2_{t} u(t,x) = c^2 \partial^2_{x} u(t,x),
\end{displaymath}
donde \(u(t,x)\) representa el desplazamiento con respecto a la posición de
equilibrio para el tiempo \(t\) en la posición \(x\) y \(c\) es una constante
real que compendia las propiedades físicas de los objetos.

La ruta usual para resolver esta ecuación es mediante el método de separación de
variables. Este consiste en separar las dos naturalezas del fénomeno, vale
decir, su naturaleza temporal y espacial. Específicamente, buscamos soluciones
de la forma
\begin{displaymath}
	u(t,x) = T(t) E(x)
\end{displaymath}
De esta forma, la ecuación de ondas se lee
\begin{displaymath}
	E(x) \partial^2_{t} T(t) = c^2 T(t) \partial^2_{x} E(x)
\end{displaymath}
Y por lo tanto
\begin{displaymath}
	\frac{\partial^2_{t} T(t)}{T(t)} 
	= 
	c^2 \frac{\partial^2_{x} E(x)}{E(x)} 
\end{displaymath}
Como las funciones y variables son independientes, ambos cocientes deben igualar
alguna constante (no necesariamente real) \(-\lambda^2\). Despejando para la componente espacial
obtenemos la ecuación de Helmholtz
\begin{displaymath}
	\partial^2_{x} E(x) + k^2 E(x) = 0
\end{displaymath}
donde \(k = \lambda / c\).

\section{Solución de la ecuación de Helmholtz}
\label{sec:helmholtz}

Para \(\psi \in \Cc^2\) un campo escalar, la ecuación
de Helmholtz se lee
\begin{equation}
\label{helmholtz}
	\nabla^2 \psi + \kappa^2 \psi = 0
,\end{equation}
donde \(\kappa\) es una constante. En coordenadas esféricas, donde para un
vector \((x,y,z)\in \R^3\) denotamor por \(r\) su componente radial, por 
\(\theta\) su componente polar y \(\varphi\) su componente azimutal 
(ver Apéndice~\ref{ap:esfericas}) nos queda:

\begin{equation}\label{helmholtz:esfericas}
	\frac{1}{r^2} \partial_{r} \left(r^2 \partial_{r} \psi\right)
	+
	\frac{1}{r^2\sin\theta} \partial_{r} \left(\sin\theta \partial_{\theta} \psi\right)
	+
	\frac{1}{r^2\sin^2\theta} \partial^2_{\varphi} \psi
	=
	-\kappa^2 \psi
,\end{equation}

Para resolver la ecuación, supondremos que \(\psi\) no tiene interdependencia
entre sus componentes, es decir, impondremos \(\psi\) una función separable de la forma
\begin{equation}
	\psi = f(r) g(\theta) h(\varphi).
\end{equation}
Reemplazando esta expresión en~\eqref{helmholtz:esfericas} nos deja

\begin{displaymath}
	gh \frac{1}{r^2} \partial_{r} \left(r^2 \partial_{r} f\right)
	+
	fh \frac{1}{r^2 \sin\theta} \partial_{\theta} \left(\sin\theta
	\partial_{\theta} g\right)
	+
	fg \frac{1}{r^2\sin^2\theta} \partial^2_{\varphi} h
	=
	-\kappa^2 fgh
\end{displaymath}
Multiplicando por \(r^2 \sin^2\theta/fgh\)

\begin{displaymath}
	\frac{\sin^2\theta}{f} \partial_{r} \left(r^2 \partial_{r} f\right)
	+
	\frac{\sin\theta}{g} \partial_{\theta} \left(\sin\theta \partial_{\theta} g\right)
	+
	\frac{1}{h} \partial^2_{\varphi} h
	=
	-{(\kappa r \sin\theta)}^2.
\end{displaymath}

Dado que el último sumando del lado izquierdo es el único dependiente de
\(\varphi\), debe ser constante para que la ecuación valga. Imponemos esto y
dejamos \(m\) ser la constante tal que

\[
	\frac{1}{h} \partial^2_{\varphi} h = -m^2.
\]

Ahora la ecuación se lee

\[
	\frac{\sin^2\theta}{f} \partial_{r} \left(r^2 \partial_{r} f\right)
	+
	\frac{\sin\theta}{g} \partial_{\theta} \left(\sin\theta \partial_{\theta} g\right)
	-
	m^2
	=
	-(\kappa r \sin\theta)^2.
\]
Dividiendo por \(\sin^2\theta\) nos deja
\[
	\frac{1}{f} \partial_{r} \left(r^2 \partial_{r} f\right)
	+
	\frac{1}{g\sin\theta} \partial_{\theta} \left(\sin\theta \partial_{\theta} g\right)
	-
	\frac{m^2}{\sin^2\theta}
	=
	-(\kappa r)^2
.\]

Similar a lo anterior, ahora los últimos dos sumandos del lado izquierdo son los
únicos que dependen de \(\theta\), por ende, deben ser constantes. Digamos que
\(n\) es la constante tal que

\begin{displaymath}
	\frac{1}{g\sin\theta} \partial_{\theta} \left(\sin\theta \partial_{\theta} g\right)
	-
	\frac{m^2}{\sin^2\theta}
	=
	-n(n+1)
.\end{displaymath}

En resumen, quedamos con tres ecuaciones desacopladas en las variables \(r\),
\(\theta\) y \(\varphi\)
\begin{alignat}{3}
\label{helmholtz:r}
	\d{r} \left(r^2 \d{r} f\right)
	+
	\left((\kappa r)^2 - n (n+1)\right) f
	&= 0
	\\
\label{helmholtz:t}
	\frac{1}{g\sin\theta} \d{\theta} \left(\sin\theta \d{\theta} g\right)
	-
	\frac{m^2}{\sin^2\theta} + n(n+1)
	&= 0
	\\
\label{helmholtz:p}
	\frac{1}{h} \partial^2_{\varphi} h + m^2
	&= 0
\end{alignat}

En~\eqref{helmholtz:r}: Si ponemos \(x = \kappa r\), nos queda una ecuación en la
forma de Bessel (esféricas)
\begin{displaymath}
\label{eq:bessel:esfericas}
	x^2 \dd{x} f + 2x \d{x} f + \left(x^2 - n(n+1)\right) f = 0
.\end{displaymath}

Esta ecuación tiene soluciones analíticas de la forma
\begin{alignat}{2}
\label{f1}
	f_1(r) &= A_{1,n,m} j_n(\kappa r) + B_{1,n,m} y_n(\kappa r)
	, \text{ o bien},\\
\label{f2}
	f_2(r) &= C_{1,n,m} h_{n}^{(1)} (\kappa r) + D_{1,n,m} h^{(2)}_{n} (\kappa r),
\end{alignat}
donde \(j_n\) es la función de Bessel esférica, \(y_n\) es la función de Bessel
esférica de segundo tipo (o función de Neumann esférica) y \(h^{(1)}_n,
h^{(2)}_{n}\) son las funciones de Hankel de primer y segundo tipo
respectivamente.

En~\eqref{helmholtz:t}: Si ponemos \(x = \cos\theta\), nos queda la ecuación de
Legendre
\begin{displaymath}
\label{eq:legendre}
	(1-x^2) \dd{x} g - 2 x \d{x} g + \left(n(n+1) - \frac{m^2}{1-x^2}\right) g
	= 0
\end{displaymath}

Cuyas soluciones son analíticas y de la forma
\begin{alignat}{2}
\label{g1}
	g_1(\theta) &= A_{2,n,m} P^m_n(\cos\theta) + B_{2,n,m} P^m_n (-\cos\theta)
	\quad&&, n\not\in \mathbb{Z}\\
\label{g2}
	g_2(\theta) &= C_{2,n,m} P^m_n(\cos\theta) + D_{2,n,m} Q^m_n (\cos\theta)
	\quad&&, n\in \mathbb{Z}
\end{alignat}

donde \(P^m_n, Q^m_n\) son las funciones de Legendre asociadas de primer y
segundo tipo respectivamente.

En~\eqref{helmholtz:p}: Las soluciones son de la forma
\begin{alignat}{2}
\label{h1}
	h_1(\varphi) &= A_{3,n,m} e^{-\i m\varphi} + B_{3,n,m} e^{\i m\varphi}\\
\label{h2}
	h_2(\varphi) &= C_{3,n,m} \cos(m\varphi) + D_{3,n,m} \sin(m\varphi)
\end{alignat}

\section{Contexto acústico}%
\label{sec:acustico}

Denotemos por \(\Omega^{\interior}\) a la esfera abierta de radio \(R>0\) en \(\R^3\), \(\Omega^{\exterior} =
\R^3\setminus \overline{\Omega}_{1}\) su exterior y \(\Gamma = \partial\Omega^{\interior}\)
su borde. Sean \(c^{\exterior/\interior}, \rho^{\exterior/\interior}\) escalares representando la
rapidez de propagación y la densidad del material respectivamente, en
\(\Omega^{\exterior/\interior}\).

Denotemos por \(p_{\inc}\in \Cc^{2}(\R^3)\) el campo de presión 
incidente y \(p_{\sca}\in \H^{2}(\overline{\Omega^{\exterior}})\) al campo
de presión dispersado por \(\Omega^{\interior}\). El campo total lo denotaremos por
\(p_{\tot} = p_{\exterior} + p_{\interior} \in \H^2(\R^3)\), donde
\(p_{\exterior/\interior} \in \H^{2}(\Omega^{\exterior/\interior})\)
denota la presión total en \(\Omega^{\exterior/\interior}\) respectivamente.

\begin{figure}[H]
\centering
\begin{tikzpicture}
	\def\localrad{1};
	\def\localdia{{2*\localrad}}
	\coordinate (Omega1) at (\localdia,0);
	\path (Omega1) -- +(45:\localdia) 
		node (Omega0) {\small \(\Omega^{\exterior},c^{\exterior},\rho^{\exterior}\)};
	\path (Omega1) -- +(0,-\localrad) node[below] (Gamma) {\(\Gamma\)};

	\draw (-1.,-\localdia) -- (-1.,\localdia);
	\draw (-.8,-\localdia) -- (-.8,\localdia);
	\draw (-.6,-\localdia) -- (-.6,\localdia);
	\draw[->] (-1.5,0) -- (.2, 0) node[below] {\scriptsize Onda plana};

	\draw[pattern=crosshatch, pattern color=black!10] (Omega1) circle (\localrad)
		node {\small \(\Omega^{\interior},c^{\interior},\rho^{\interior}\)};

	\draw[-Triangle,thick] ($(Omega1) + (10:\localrad)$) -- +(10:\localrad) 
		node[below] {\(\hat n\)};
\end{tikzpicture}
\caption{Contexto acústico}
\end{figure}

En lo que sigue, asumiremos que \(c^{\exterior}, \rho^{\exterior}, c^{\interior}, \rho^{\interior}\)
son constantes y que tenemos una onda plana (modelada por \(p_{\inc}\)) propagándose en el espacio
con frecuencia \(\omega\). Se sigue que
\begin{subequations}\label{eq:helmholtz}
\begin{equation}
\label{eq:helmholtz:pinc}
	\nabla^2 p_{\inc} + \kappa^2_{\exterior} p_{\inc} = 0
	,\text{ en }\R^3
\end{equation}
donde \(\kappa_{\exterior} = \omega/c^{\exterior}\) es el número de onda. Por otro lado,
\(p_{\tot}\) satisface la ecuación de Helmholtz en ambas regiones
\begin{equation}
\label{eq:helmholtz:ptot}
	\nabla^2 p_{\tot} + \kappa^2_{\exterior/\interior} p_{\tot} = 0
	, \text{ en } \Omega^{\exterior/\interior}
\end{equation}
Adicionalmente, la onda dispersada debe cumplir la \textit{condición de radiación de Sommerfeld}
\begin{equation}
\label{sommerfeld}
	\lim_{r\to \infty}
	r
	\left(
		\partial_r p_{\sca} - \i\kappa_{\exterior} p_{\sca}
	\right)
	= 0
	,\text{ donde } r = \norm{\x}.
\end{equation}
\end{subequations}

Para una solución a la ecuación de onda es de la forma 
\begin{displaymath}
	u(t,\x) = p(\x) e^{-\i \omega t}
\end{displaymath}
Una onda plana \(p_{\textrm{plana}}\) se puede modelar como una 
combinación lineal de ondas esféricas de la siguiente forma:
\[
	e^{\i \kappa_{\exterior}\, \x \cdot \dir}
	=
	\sum_{n=0}^{\infty}
	(2n+1) \i^{n}
	j_{n} (\kappa_{\exterior} \norm{\x})
	P_{n} (\hat{\x} \cdot \dir)
	,\qquad
	\hat{\x} = \x/\norm{\x}
,\]
donde \(\x,\dir \in \R^3\) describen una posición en el espacio y
la dirección de propagación (unitaria) de la onda respectivamente. La constante
\(\kappa_{\exterior}\) es el número de onda en \(\Omega^{\exterior}\).

Dado que la esfera es simétrica, la dirección de onda incidente es indiferente,
por lo tanto, tomaremos la dirección \(\dir = (0,0,1)\). Es decir, la onda
incidente viaja a paralela al eje \(z\) hacia \(+z\). En símbolos, la onda 
incidente en coordenadas esféricas se lee 
\[
	e^{\i \kappa_{\exterior} r \cos\theta}
	=
	\sum_{n=0}^{\infty}
	(2n+1) \i^{n}
	j_{n} (\kappa_{\exterior} r)
	P_{n} (\cos\theta).
\]
Esto nos da una simetría azimutal.

\section{Soluciones Analíticas}

\subsection{Solución al Problema Exterior}\label{ssec:a:exterior}

Buscaremos soluciones para la dispersión en \(\Omega^{\exterior}\). En particular,
\(p_{\tot} = p_{\exterior} = p_{\sca} + p_{\inc}\).

\subsubsection{Condiciones de borde de Dirichlet}
\label{ssec:a:exterior:dirichlet}

\noindent La condición de borde adicional es:
\begin{align}\label{bc:dirichlet}
	p_{\exterior} &= 0, \text{ en } \Gamma\\[-5pt]
	\intertext{Equivalentemente}
	p_{\inc} &= -p_{\sca}, \text{ en } \Gamma
\end{align}

Recordar que tanto \(p_{\inc}\) como \(p_{\sca}\) satisfacen la ecuación de
Helmholtz en \(\Omega^{\exterior}\). En particular, asumiremos que \(p_{\sca}\) es una
superposición de soluciones de la forma \(fgh\) donde \(f,g\) y \(h\) representan la componente
radial (Ecuaciones~\eqref{f1},\eqref{f2}), polar (Ecuaciones~\eqref{g1},\eqref{g2}) y azimutal
(Ecuaciones~\eqref{h1},\eqref{h2}) respectivamente, i.e.
\begin{displaymath}
	p_{\sca} = \sum_{n=0}^{\infty} b_n f_n(r) g_n(\theta) h_n(\varphi).
\end{displaymath}

Dada la expresión de \(p_{\inc}\) y con el fin de tener
una base supondremos que el componente \(g_n\) es de la forma~\eqref{g2} 
(i.e. \(g_2\) puesto que \(n\) es entero) con el coeficiente \(D_{2,n,0}\) nulo. 
En resumen,
\begin{equation}
\label{psca:g}
	g_n(\theta)
	=
	C_{2,n,0} P_{n} (\cos\theta)
\end{equation}
De esta forma, el componente azimutal debe ser constante (pues \(m=0\)).
Por lo tanto:
\begin{equation}\label{psca:h}
	h_n(\varphi)
	=
	C_{3,n,0}
\end{equation}
Nótese que esto es consistente con la simetría azimutal antes mencionada, pues si
no fuese constante se rompería dicha simetría.

Por último, dado que \(p_{\sca}\) debe satisfacer la condición de radiación de
Sommerfeld~\eqref{sommerfeld}, el componente radial será de la forma~\eqref{f2}
con \(D_{1,n,0} = 0\) pues \(p_{\sca}\) debe ser una onda viajera hacia infinito (ver
Apéndice~\ref{ap:fn especiales}). De esta forma, la ecuación que
describe a la onda dispersada en \(\R^3\setminus\Omega^{\interior}\) es de la
forma
\begin{equation}\label{psca:forma}
	p_{\sca}
	=
	\sum_{n=0}^{\infty}
	\underbrace{C_{1,n,0} C_{2,n,0} C_{3,n,0}}_{\eqqcolon \alpha_n}
		h^{(1)}_{n}(\kappa_{\exterior} r)
		P_{n}(\cos\theta)
.\end{equation}

\noindent Por~\eqref{bc:dirichlet} y usando que los polinomios de Legendre son
ortogonales, los coeficientes de \(p_{\sca}\) son
\begin{equation}\label{ext:psca:coef:dirichlet}
	\alpha_n 
	= 
	-\frac
	{
		(2n+1)\i^n j_n(\kappa_{\exterior} R)
	}
	{
		h^{(1)}_{n} (\kappa_{\exterior} R)
	}
\end{equation}

\subsubsection{Condiciones de borde de Neumann}\label{sssec:a:exterior:neumann}

\noindent Aquí las condiciones de borde son:
\begin{align}
\label{bc:neumann}
	\nabla p_{\exterior}\cdot \n &= 0, \text{ en }
	\Gamma\\[-1em]
	\intertext{Equivalentemente}
	\nabla p_{\inc}\cdot \n &= -\nabla p_{\sca}\cdot \n, \text{ en }
	\Gamma
\end{align}
donde \(\n\) es un vector normal exterior en el punto \(\x\in\Gamma\). En coordenadas
esféricas queda
\begin{equation}\label{n}
	\n = (\sin\theta\cos\varphi, \sin\theta\sin\varphi, \cos\theta).
\end{equation}
donde \(\x = (R\sin\theta\cos\varphi, R\sin\theta\sin\varphi, R\cos\theta)\).
De esta forma, el operador \(\nabla\cdot\n\) se lee:
\begin{equation}
\label{nabla.n}
	\nabla\cdot\n = \partial_{r}
\end{equation}

\noindent Así, la condición~\eqref{bc:neumann} en coordenadas esféricas
se traduce a la siguiente expresión:
\begin{equation}\label{neumann:esfericas}
	\partial_{r} p_{\inc} = - \partial_{r} p_{\sca}, \text{ en } \Gamma.
\end{equation}

\noindent Aquí también buscamos expresar \(p_{\sca}\) como una superposición de
soluciones. Dado que
solo estamos derivando el componente radial, la elección de \(g_n\) y \(h_n\) se
mantiene. Más aún, por~\eqref{sommerfeld} la preferencia de \(f_n\) también se
conserva. Por lo tanto, en \(\R^3\setminus\Omega^{\interior}\) la expresión para
la onda dispersada es:
\begin{equation}
\label{ext:psca:neumann:forma}
	p_{\sca}
	=
	\sum_{n=0}^{\infty}
	\underbrace{C_{1,n,0} C_{2,n,0} C_{3,n,0}}_{\eqqcolon \beta_n}
		h^{(1)}_{n}(\kappa_{\exterior} r)
		P_{n}(\cos\theta)
\end{equation}
donde, por~\eqref{bc:neumann} y la ortogonalidad de los polinomios de Legendre
se tiene que:
\begin{equation}
\label{ext:psca:neumann:coef}
	\beta_n
	=
	-\frac
	{
		(2n+1) \i^n \partial_{r} j_n(\kappa_{\exterior} R)
	}
	{
		\partial_r h_n^{(1)}(\kappa_{\exterior} R)
	}
\end{equation}
Notar que \(j_n\) es una función univariada, por lo que \(\partial_r = \d{r}\).

\subsection{Solución al Problema de Transmisión}
\label{sec:a:transmission}

El problema de transmisión consiste en agregar a~\eqref{eq:helmholtz} las
siguientes condiciones:
\begin{subequations}\label{eq:transmision}
\begin{alignat}{2}
\label{eq:transmision:D}
	p_{\exterior} &= p_{\interior}
	&\quad&, \text{ en } \Gamma\\
\label{eq:transmision:N}
	\frac{1}{\rho^{\exterior}} \nabla p_{\exterior} \cdot \n
	&=
	\frac{1}{\rho^{\interior}} \nabla p_{\interior} \cdot \n
	&\quad&, \text{ en } \Gamma
\end{alignat}
\end{subequations}

Por lo desarrollado en la Sección~\ref{sec:helmholtz} sabemos que la onda
disperada e interior son de la forma:
\begin{alignat}{2}
	p_{\sca}(r,\theta,\varphi)
	&=
	\sum_{n=0}^{\infty} f^{\sca}_{n}(r) g^{\sca}_{n}(\theta) h^{\sca}_{n}(\varphi)
	\\
	p_{\interior}(r,\theta,\varphi)
	&=
	\sum_{n=0}^{\infty} f^{\interior}_{n}(r) g^{\interior}_{n}(\theta) h^{\interior}_{n}(\varphi)
\end{alignat}

Dado que \(p_{\sca}\) satisface la condición de radiación de
Sommerfeld~\eqref{sommerfeld}, su componente radial debe ser de la forma
\begin{equation}
	f^{\sca}_{n}(r)
	=
	C_{1,m,n} h^{(1)}_{n}(\kappa_{\exterior} r)
\end{equation}
pues necesitamos ondas viajeras en sentido positivo.

De manera análoga, el componente radial de la onda interior debe ser de la forma
\begin{equation}
	f^{\interior}_{n}(r)
	=
	A_{1,n,m} j_{n}(\kappa_{\interior} r)
\end{equation}
que corresponde con ondas estacionarias.

Con el fin de tener una base para comparar, tomaremos \(g^{\sca/\interior}\) de
la forma \(g^{\sca/\interior}_{n} = C^{\sca}_{2,n,0} P_{n} +
D^{\interior}_{2,n,0} Q_{n}\) y pondremos \(D^{\sca/\interior}_{2,n,0} = 0\). 
En consecuencia, \(h^{\sca/\interior}_{n} = C_{3,n,0}^{\sca/\interior}\) una
constante\footnote{Esto es consistente con la simetría azimutal.}.
De esta forma, la onda dispersada e interior son de la forma:
\begin{alignat}{2}
	p_{\sca}(r,\theta,\varphi)
	&=
	\sum_{n=0}^{\infty}
	\left[
		C_{1,n,0} C^{\sca}_{2,n,0} C^{\sca}_{3,n,0} h^{(1)}_{n}(\kappa_{\exterior} r)
	\right] P_{n}(\cos\theta)
	\\
	p_{\interior}(r,\theta,\varphi)
	&=
	\sum_{n=0}^{\infty}
	\left[
		A_{1,n,0} C^{\interior}_{2,n,0} C^{\interior}_{3,n,0} j_{n}(\kappa_{\interior} r)
	\right] P_{n}(\cos\theta),
\end{alignat}

Denotemos por \(\sigma_n\) al los coeficientes de \(p_{\sca}\) y por 
\(\iota_n\) al los coeficientes de \(p_{\inc}\). De esta forma y 
por~\eqref{eq:transmision} tenemos que
\begin{alignat}{2}
	\sum_{n=0}^{\infty}
	\left[
		(2n + 1) \i^n j_{n}(\kappa_{\exterior} R)
		+
		\sigma_{n} h^{(1)}_{n}(\kappa_{\exterior} R)
	\right] P_{n}(\cos\theta)
	&=
	\sum_{n=0}^{\infty}
		\iota_{n} j_{n} (\kappa_{\interior} R)
		P_{n}(\cos\theta)
	\\
	\sum_{n=0}^{\infty}
	\frac{\rho^{\interior}}{\rho_{\exterior}}\kappa_{\exterior}
	\left[
		(2n + 1) \i^n \partial_{r} j_{n}(\kappa_{\exterior} R)
		+
		\sigma_{n} \partial_{r} h^{(1)}_{n}(\kappa_{\exterior} R)
	\right] P_{n}(\cos\theta)
	&=
	\sum_{n=0}^{\infty}
		\kappa_{\interior}
		\iota_{n} \partial_{r} j_{n} (\kappa_{\interior} R)
		P_{n}(\cos\theta)
\end{alignat}
Donde usamos que \(\partial_{\n} = \partial_{r}\) en coordenadas esféricas y
\(\partial_{r} P_{n}(\cos\theta) = 0\).
Luego, usando la ortogonalidad de los polinomios de Legendre obtenemos el 
sistema lineal
\begin{alignat}{2}
	(2n+1)\i^n j_{n}(\kappa_{\exterior} R)
	+
	\sigma_{n} h^{(1)}_{n}(\kappa_{\exterior} R)
	&=
	\iota_{n} j_{n}(\kappa_{\interior} R)
	\\
	\frac{\rho^{\interior}}{\rho^{\exterior}} \kappa_{\exterior}
	\left[
		(2n+1)\i^n \partial_{r} j_{n}(\kappa_{\exterior} R)
		+
		\sigma_{n} \partial_{r} h^{(1)}_{n}(\kappa_{\exterior} R)
	\right]
	&=
	\kappa_{\interior} \iota_{n} \partial_{r} j_{n}(\kappa_{\interior} R)
\end{alignat}
para cada \(n \ge 0\). 

Dejemos a \(\rho \coloneqq \rho^{\interior}/\rho^{\exterior}\) y \(\kappa
\coloneqq \kappa_{\exterior}/\kappa_{\interior}\). Luego, 
resolviendo el sistema se obtiene:
\begin{alignat}{2}
	\sigma_{n}
	&=
	\frac
	{%
		(2n+1) \i^n
		\left[
			j_n(\kappa_{\exterior} R)
			j'_{n}(\kappa_{\interior} R)
			-
			\rho \kappa
			j_{n}(\kappa_{\interior} R)
			j'_{n}(\kappa_{\exterior} R)
		\right]
	}
	{%
		\rho \kappa
		j_{n}(\kappa_{\interior} R)
		h^{(1)'}_{n}(\kappa_{\exterior} R)
		-
		j'_{n}(\kappa_{\interior} R)
		h^{(1)}_{n}(\kappa_{\exterior} R)
	}
	\\
	\iota_{n}
	&=
	\frac
		{(2n + 1) \i^n j_n(\kappa_{\exterior} R) + \sigma_{n} h^{(1)}_{n} (\kappa_{\exterior} R)}
		{j_{n}(\kappa_{\interior} R)}
\end{alignat}

\section{Método de Elementos de Frontera}

El método de elementos de frontera (\textit{BEM} por sus siglás en inglés)
consiste en reformular un problema volumétrico (de ecuaciones
diferenciales parciales) como un problema integral de potenciales en la
frontera.

De manera más específica, las condiciones de frontera se representarán mediante
condiciones a los operadores de traza. Denotamos al operador de traza de Dirichlet
como \(\traceD^{\exterior/\interior}\), mientras que para el operador de traza de Neumann
usamos la notación \(\traceN^{\exterior/\interior}\). Los operadores se definen como:
\begin{alignat}{2}
\label{op:trace_dirichlet}
	(\traceD^{\exterior/\interior} \psi)(x)
	&\coloneqq \lim_{\Omega^{\exterior/\interior}\ni y\to x} \psi(y)
	\quad& x\in \Gamma
	\\
\label{op:trace:neumann}
	(\traceN^{\exterior/\interior} \psi)(x)
	&\coloneqq \lim_{\Omega^{\exterior/\interior}\ni y\to x}
	\partial_{\n(x)} \psi(y)
	\quad& x\in \Gamma
\end{alignat}

Por otro lado, los potenciales que usaremos son de la forma:
\begin{alignat}{2}
\label{op:pot:sl}
	(\Slpot_{\kappa} \psi)(x)
	&\coloneqq
	\int_{\Gamma} G_{\kappa}(x,y)
	\psi(y)\,\textrm{d}\Gamma(y)
	\quad& x\in \Omega^{\interior} \cup \Omega^{\exterior}
	\\
\label{op:pot:dl}
	(\Dlpot_{\kappa} \psi)(x)
	&\coloneqq
	\int_{\Gamma} \partial_{\n(y)} G_{\kappa}(x,y) \psi(y)\, \textrm{d}\Gamma(y)
	\quad& x\in \Omega^{\interior} \cup \Omega^{\exterior}
\end{alignat}
donde \(G_{\kappa}\) es la función de Green para el caso acústico dada por:
\begin{equation}\label{green}
	G_{\kappa} (x,y) = \frac{e^{\i \kappa \abs{x-y}}}{4\pi \abs{x-y}},\quad x\ne y.
\end{equation}
Decimos que~\eqref{op:pot:sl} es el \textit{operador potencial
de primera capa} y~\eqref{op:pot:dl} es el \textit{operador potencial de segunda capa}.

Para la soluciones \(p\) a la ecuación de Helmholtz con un número de onda
\(\kappa\) dado usamos el ansatz:
\begin{subequations}\label{eq:ansatz}
\begin{alignat}{3}
	p(x) &= (\Slpot f)(x) - (\Dlpot g )(x)
	&,\quad x\in \Omega^{\interior}\cup\Omega^{\exterior}\\
\intertext{donde}
	f &= \traceN^{\,\interior} p - \traceN^{\,\exterior} p\\
	g &= \traceD^{\,\interior} p - \traceD^{\,\exterior} p
\end{alignat}
\end{subequations}

Supondremos que \(\Gamma\) es lo suficientemente suave como para que los
siguiente operadores de frontera estén bien definidos casi en todas partes.
\begin{subequations}\label{operadores integrales de frontera}
\begin{alignat}{4}
\label{op:int:sl}
	(\Slop f)(x) &\coloneqq
	\int_{\Gamma} G(x,y) f(y) \textrm{dy}
	&\quad&, x\in \Gamma\\
\label{op:int:dl}
	(\Dlop g )(x) &\coloneqq
	\int_{\Gamma} \partial_{\n(y)} G(x,y) g(y) \textrm{dy}
	&\quad&, x\in \Gamma\\
\label{op:int:adl}
	(\Adlop f)(x) &\coloneqq
	\int_{\Gamma} \partial_{\n(x)} G(x,y) f(y) \textrm{dy}
	&\quad&, x\in \Gamma\\
\label{op:int:h}
	(\Hop g )(x) &\coloneqq 
	-\partial_{\n(x)} \int_{\Gamma} \partial_{\n(y)} G(x,y) g(y)\textrm{dy}
	&\quad&, x\in \Gamma
\end{alignat}
\end{subequations}
Así como sus relaciones de salto
\begin{subequations}\label{relaciones de salto}
\begin{alignat}{4}
	\Slop f
	&= \traceD^{\,\exterior} \left(\Slpot f\right)
	&&= \traceD^{\,\interior} \left(\Slpot f\right)
	\\
	\Dlop g
	&= \traceD^{\,\exterior} \left(\Dlpot g\right) - \frac{1}{2}g
	&&= \traceD^{\,\interior} \left(\Dlpot g\right) + \frac{1}{2}g
	\\
	\Adlop f
	&= \traceN^{\,\exterior} \left(\Slpot f\right) + \frac{1}{2}f
	&&= \traceN^{\,\interior} \left(\Slpot f\right) - \frac{1}{2}f
	\\
	\Hop g
	&= -\traceN^{\,\exterior} \left( \Dlpot g \right)
	&&= -\traceN^{\,\interior} \left( \Dlpot g \right)
\end{alignat}
\end{subequations}

Los operadores integrales de frontera \(\Slop\), \(\Dlop\), \(\Adlop\) y \(\Hop\) son llamados 
de \textit{primera capa, segunda capa, adjunto de segunda capa} e
\textit{hipersingular} respectivamente. Nótese que los operadores dependen del
número de onda. 

\subsection{Soluciones al Problema Exterior}
\label{ssec:bem:exterior}

Como en el caso analítico, buscaremos soluciones para el campo acústico en
\(\Omega^{\exterior}\), en particular, \(p_{\interior} = 0\). Una consecuencia
inmediata es que \(p_{\tot} = p_{\exterior} = p_{\sca} + p_{\inc}\).

\subsubsection{Condiciones de Borde de Dirichlet}
\label{sssec:bem:exterior:dirichlet}

\noindent Estamos pidiendo la condición
\begin{equation}\label{bem:condicion dirichlet}
	\traceD^{\,\exterior} p_{\tot} = 0, \quad \text{ en } \Gamma.
\end{equation}
Dado que conocemos \(p_{\inc}\), usaremos~\eqref{eq:ansatz} como ansatz
para \(p_{\sca}\). Se sigue (usando la continuidad de la onda
incidente y las condiciones de borde e interior) que
\begin{alignat*}{2}
	g
	&=
	\traceD^{\,\interior} p_{\sca} - \traceD^{\,\exterior} p_{\sca}
	&=
	\traceD^{\,\interior} p_{\exterior}
	-
	\traceD^{\,\exterior} p_{\exterior}
	+
	\traceD^{\,\exterior} p_{\inc} - \gamma_D^{\interior} p_{\inc}
	&=
	0.
	\\
	f
	&=
	\traceN^{\,\interior} p_{\sca} - \traceN^{\,\exterior} p_{\sca}
	&=
	\traceN^{\,\interior} p_{\exterior} - \traceN^{\,\exterior} p_{\exterior}
	+
	\traceN^{\,\exterior} p_{\inc} - \traceN^{\,\interior} p_{\inc}
	&=
	- \traceN^{\,\exterior} p_{\exterior}.
\end{alignat*}

De esta forma y poniendo \(\psi = \traceN^{\exterior} p_{\exterior}\), la onda dispersada se lee
\begin{align}\label{bem:psca:dirichlet}
	p_{\sca}(x)
	&=
	\left[ \Slpot f \right] (x)
	=
	-\left[ \Slpot \psi \right](x)
	,\quad x\in \Omega^{\exterior}.
\end{align}
Por lo que si encontramos \(\psi\), obtenemos una expresión para \(p_{\sca}\). 

Aplicando \(\traceD^{\,\exterior}\) a~\eqref{bem:psca:dirichlet} y usando las relaciones
de salto nos queda que \( \traceD^{\,\exterior} p_{\sca} = -\Slop \psi \).
De manera similar, aplicando \(\traceN^{\,\exterior}\) obtenemos
\(\traceN^{\,\exterior} p_{\sca} = -\Adlop \psi + \frac{1}{2} f\).
Descomponiendo \(p_{\sca} = p_{\exterior} - p_{\inc}\) obtenemos las ecuaciones
\begin{displaymath}
	p_{\inc} = \Slop \psi
	\hspace{1cm}\text{y}\hspace{1cm}
	\partial_{\n} p_{\inc} = \Adlop \psi + \frac{1}{2} \psi
\end{displaymath}

\noindent Combinando las ecuaciones bajo un parámetro \(\eta \in \C\) nos queda
\begin{equation}
	\left(\frac{1}{2} \Idop + \Adlop + \eta \Slop \right)\psi
	=
	\partial_{\n} p_{\inc} + \eta p_{\inc}
\end{equation}
Como \(p_{\inc}\) es conocida, podemos resolver para \(\psi\). 

\subsubsection{Condiciones de Borde de Neumann}
\label{sssec:bem:exterior:neumann}

\noindent Ahora la condición es
\begin{equation}\label{bem:neumann}
	\traceN^{\,\exterior} p_{\tot} = 0, \text{ en } \Gamma.
\end{equation}

\noindent Como antes, esto caracteriza a las funciones \(f\) y \(g\) de la siguiente
manera:
\begin{alignat*}{2}
	f &=
	\traceN^{\,\interior} p_{\sca} - \traceN^{\,\exterior} p_{\sca}
	&=
	\traceN^{\,\interior} p_{\exterior} - \traceN^{\,\exterior} p_{\exterior}
	+
	\traceN^{\,\exterior} p_{\inc} - \traceN^{\,\interior} p_{\inc}
	&= 0\\
	g &=
	\traceD^{\,\interior} p_{\sca} - \traceD^{\,\exterior} p_{\sca}
	&=
	\traceD^{\,\interior} p_{\exterior} - \traceD^{\,\exterior} p_{\exterior}
	+
	\traceD^{\,\exterior} p_{\inc} - \traceD^{\,\interior} p_{\inc}
	&=
	-\traceD^{\,\exterior} p_{\exterior}
\end{alignat*}

Definamos \(\phi \coloneqq \traceD^{\,\exterior} p_{\exterior} = -g\). Luego, la onda
dispersada se lee:
\begin{equation}
	p_{\sca} 
	= \Dlpot (\traceD^{\,\exterior} p_{\exterior})
	= \Dlpot \phi
	,\textrm{ en } \Omega^{\exterior}.
\end{equation}

Por las relaciones de salto tenemos que
\(\traceN^{\,\exterior} p_{\sca} = \Hop \phi\) y \(\traceD^{\,\exterior}
p_{\sca} = \Dlop \phi + (1/2) \phi\). Descomponemos \(p_{\sca} = p_{\exterior} -
p_{\inc}\). Usando la continuidad de \(p_{\inc}\) y la condición de borde obtenemos
las siguientes relaciones: 
\[
	p_{\inc} = -\left(\Dlop - \frac{1}{2} \Idop \right) \phi
	\hspace{1cm}\text{y}\hspace{1cm}
	\partial_{\n} p_{\inc} = -D\phi
.\]

\noindent Combinando las ecuaciones bajo un parámetro \(\eta \in \C\) nos queda
\begin{equation}
	\left(\frac{1}{2} \Idop - \Dlop - \eta \Hop \right) \phi
	=
	p_{\inc} + \eta \partial_{\n} p_{\inc}
\end{equation}
De nuevo, \(p_{\inc}\) es conocida, por lo que basta resolver para \(\phi\). 

\subsection{Solución al Problema de Transmisión}
\label{ssec:bem:transmission}

Consiste en agregar las condiciones:
\begin{subequations}\label{bem:tp}
	\begin{alignat}{2}
		\gamma^{\exterior}_{D} p_{\tot}
		&= 
		\gamma^{\interior}_{D} p_{\tot} 
		&\quad&, \textrm{ en } \Gamma
		\\
		\frac{1}{\rho^{\exterior}}
		\gamma^{\exterior}_{N} p_{\tot}
		&= 
		\frac{1}{\rho^{\interior}}
		\gamma^{\interior}_{N} p_{\tot} 
		&\quad&, \textrm{ en } \Gamma
	\end{alignat}
\end{subequations}

Equivalentemente, podemos expresar estas condiciones introduciendo el
\textit{operador traza de Cauchy}:
\begin{equation} \label{op:trace:cauchy}
	\gamma^{\exterior/\interior}
	=
	\begin{bmatrix}
		\traceD^{\exterior/\interior} \\[5pt]
		\traceN^{\exterior/\interior}
	\end{bmatrix}.
\end{equation}

De esta forma, el problema de transmisión queda:
\begin{equation}
	\Rho
	\gamma^{\exterior} p_{\tot} 
	=
	\gamma^{\interior} p_{\tot}
	\quad, \textrm{ en } \Gamma
\end{equation}
donde 
\begin{equation}
	\Rho
	= 
	\begin{bmatrix}
		1 & 0\\
		0 & \rho^{\interior}/\rho^{\exterior}	
	\end{bmatrix}
\end{equation}

Para la presión acústica total dispersada usaremos el ansatz~\eqref{eq:ansatz}
con \(p_{\tot}\). 

Por las relaciones de salto se tiene que:
\begin{alignat}{2}
	\gamma^{\exterior} p_{\sca}
	&=
	\gamma^{\exterior} \left[ \Slpot f - \Dlpot g \right]
	&&=
	\left( A^{\exterior} - \frac{1}{2} \Idop \right) 
	\begin{pmatrix}
		g \\ f
	\end{pmatrix}
	\\
	\gamma^{\interior} p_{\interior}
	&=
	\gamma^{\interior} \left[ \Slpot f - \Dlpot g \right]
	&&=
	\left( A^{\interior} + \frac{1}{2} \Idop \right) 
	\begin{pmatrix}
		g \\ f
	\end{pmatrix}
\end{alignat}
donde \(A^{\exterior/\interior}\) es el \textit{operador multitraza} dado por:
\begin{equation}
	A^{\exterior/\interior} = 
	\begin{bmatrix}
		-\Dlop & \Slop \\
		\Hop & \Adlop
	\end{bmatrix}.
\end{equation}
Notar que los operadores dependen del número de onda \(\kappa\), por lo que en general
\(A^{\exterior} \ne A^{\interior}\).  

\begin{subequations}
Para las trazas exteriores, usaremos los potenciales de salto quitando la
contribución de los términos interiores, es decir \(f = -\traceN^{\exterior}
p_{\tot}\) y \(g = -\traceD^{\exterior} p_{\tot}\). De esta forma, se tiene que:
\begin{equation}\label{bem:tp:relacion_ext}
	\gamma^{\exterior} p_{\sca}
	=
	\left( \frac{1}{2} \Idop - A^{\exterior} \right)
	\gamma^{\exterior} p_{\tot}.
\end{equation}

Análogamente para las trazas interiores ignoraremos las constribuciones del
exterior, es decir, \(f = \traceN^{\interior} p_{\tot}\) y \(g =
\traceD^{\interior} p_{\tot}\). Teniendose que:
\begin{equation}\label{bem:tp:relacion_int}
	\gamma^{\interior} p_{\interior}
	=
	\left( \frac{1}{2} \Idop + A^{\interior} \right) 
	\gamma^{\interior} p_{\tot}.
\end{equation}
\end{subequations}

Notése que en el interior \(p_{\interior} = p_{\tot}\), por lo que aplicando las
condiciones de borde~\eqref{bem:tp} a~\eqref{bem:tp:relacion_int} se obtiene:
\begin{equation}
	P \gamma^{\exterior} p_{\tot}
	=
	\left( \frac{1}{2} \Idop + A^{\interior} \right) 
	P \gamma^{\exterior} p_{\tot}.
\end{equation}
Definiendo \(\tilde{A}^{\interior} = P^{-1} A^{\interior} P\) nos queda que
\begin{equation}
	\gamma^{\exterior} p_{\tot}
	=
	\left( \frac{1}{2} \Idop + \tilde{A}^{\interior} \right) 
	\gamma^{\exterior} p_{\tot}.
\end{equation}
Restando~\eqref{bem:tp:relacion_ext} nos da la ecuación
\begin{equation}
	\gamma^{\exterior} p_{\inc}
	=
	\left( A^{\exterior} + \tilde{A}^{\interior} \right) 
	\gamma^{\exterior} p_{\tot}.
\end{equation}

\newgeometry{lmargin=2em,rmargin=2em}
\section{Resumen}

\setlength{\arrayrulewidth}{.25mm}
\setlength{\tabcolsep}{.5cm}
\renewcommand{\arraystretch}{2}

\begin{table}[H]
\centering
\begin{tabular}{ |p{1cm}|p{4cm}|p{2cm}|p{4cm}|  }
	\hline
	\multicolumn{4}{|c|}{Problema Exterior} \\
	\hline
	Variable & Forma & \multicolumn{2}{|c|}{Coeficientes} \\
	\hline
	\multirow{2}{2cm}{ \(p_{\sca}\) }
	& 
	\multirow{2}{2cm}{ \(\sum_{n=0}^{\infty} a_n h^{(1)}_{n}(\kappa_{\exterior} r) P_{n}(\cos\theta)\) }
	&
	Dirichlet& \(a_n = -\dfrac{(2n+1)\i^n j_n(\kappa R)}{h^{(1)}_{n}(\kappa R)}\) \\[1em]
	\cline{3-4}
	&&
	Neumann& \( a_n = -\dfrac{(2n+1) \i^n j_n'(\kappa R)}{h^{(1)'}_{n}(\kappa R)}\)
	\\[1em]
	\hline
	\multicolumn{4}{|c|}{Problema de Transmisión} \\
	\hline
	Variable & Forma & \multicolumn{2}{|c|}{Coeficiente} \\
	\hline
	\(p_{\sca}\)
	& 
	\(\sum_{n=0}^{\infty} a_n h^{(1)}_{n}(\kappa_{\exterior} r) P_{n}(\cos\theta)\)
	&
	\multicolumn{2}{|c|}{\(
	a_{n} =
	(2n+1) \i^n
	\dfrac
	{%
		\left[
			\rho_{\exterior} \kappa_{\interior}
			j_n(\kappa_{\exterior} R)
			j'_{n}(\kappa_{\interior} R)
			-
			\rho_{\interior} \kappa_{\exterior}
			j_{n}(\kappa_{\interior} R)
			j'_{n}(\kappa_{\exterior} R)
		\right]
	}
	{%
		\rho_{\interior} \kappa_{\exterior}
		j_{n}(\kappa_{\interior} R)
		h'^{(1)}_{n}(\kappa_{\exterior} R)
		-
		\rho_{\exterior} \kappa_{\interior}
		j'_{n}(\kappa_{\interior} R)
		h^{(1)}_{n}(\kappa_{\exterior} R)
	}\)
	} \\[1em]
	\hline
	\(p_{\interior}\)
	&
	\( \sum_{n=0}^{\infty} b_n j_{n}(\kappa_{\interior} r) P_{n}(\cos\theta) \)
	&
	\multicolumn{2}{|c|}{\(
	b_n =
	\dfrac
	{
		(2n + 1) \i^n j_n(\kappa_{\exterior} R) 
		+ 
		a_{n} h^{(1)}_{n} (\kappa_{\exterior} R)
	}
	{
		j_{n}(\kappa_{\interior} R)
	}
	\)}\\[1em]
	\hline
\end{tabular}
\caption{Fórmulas para soluciones analíticas}
\end{table}

\appendix

\restoregeometry

\section{Notación}\label{ap:notacion}

El operador del (o nabla) \(\nabla\) está definido por
\begin{equation}
	\nabla = \begin{pmatrix} \partial_x \\ \partial_y \\ \partial_z \end{pmatrix}
,\end{equation}
donde \(\partial_{\bullet}\) denotará a la derivada parcial con respecto a la variable
\(\bullet\). Para un campo escalar, el \textit{Laplaciano} está definido como
\begin{equation}
	\nabla^2 = \partial^2_x + \partial^2_y + \partial^2_z.
\end{equation}
Por otro lado, se recomienda ver el Apéndice~\ref{ap:esfericas} si no se está familiarizado con las
coordenadas esféricas.

Para \(X\subset \R^3\) denotamos por \(X'\) al conjunto se sus puntos
límite, \(X^{\circ}\) a sus puntos interiores, \(\overline{X} = X \cup X'\) a su
clausura y por \(\partial X\) su frontera.

Decimos que \(\Cc^k(X)\) son las funciones \(k\)-veces diferenciable con la
\(k\)-ésima derivada continua sobre \(X\). Por otro lado, \(L^p(X)\) denota el espacio de
las funciones medibles sobre \(X\) con norma \(\norm{\cdotp}_{p}\). Por último,
usamos la notación \(\H^{k}(X)\) para denotar el espacio de Sobolev de orden
\(k\) asociado a \(L^2\). Si \(X\) se omite se asume que es \(\R^3\) o bien el dominio
correspondiente en caso de que ya esté definido. En todos los casos las funciones toman
valores en \(\C\).

\section{Coordenadas esféricas}%
\label{ap:esfericas}

Tomando las coordenadas usuales como punto de partida, el sistema de referencia
esférico expresa el espacio tridimensional con respecto a tres
datos: La distancia al origen (\(r\)), el ángulo con respecto al eje \(z\) y el
ángulo de la proyección en el plano \(xy\) con el eje \(x\). De manera más
precisa, las relaciones son las siguientes

\begin{equation}
\begin{array}{rl}
	x &= r \sin(\theta) \cos(\varphi) \\
	y &= r \sin(\theta) \sin(\varphi) \\
	z &= r \cos(\theta)
\end{array}
\,\longleftrightarrow\,
\begin{array}{rl}
	r       &= \sqrt{x^2 + y^2 + z^2}\\
	\theta  &= \arctan \left(y/x\right)\\
	\varphi &= \arccos \left(z/r\right)
\end{array}
\hspace{.5cm}\text{con}\hspace{.5cm}
\begin{array}{rl}
	r &\in [0,\infty),\\
	\theta &\in [0,\pi],\\
	\varphi &\in [0, 2\pi)
\end{array}
\end{equation}
\subsection{Cambio de coordenadas y derivadas}%
\label{ap:esfericas:nabla}

Lidiaremos con funciones \(f\colon \R^3 \to \C\) expresadas
en el sistema de referencia esférico. Dado que operadores como \(\nabla\) están
expresados en términos de las coordenadas rectángulares, necesitamos una
expresión de \(\nabla\) para funciones que usan coordenadas esféricas. En
resumen, \(f\) recibe argumentos en formato \((r,\theta,\varphi)\) y nosotros
tenemos argumentos del tipo \((x,y,z)\), así que necesitamos una forma de
comunicar ambos sistemas, y con ello sus operadores asociados.

Consideremos \(T\colon \R^3 \to \R^3\) la función que transforma
las coordenadas rectangulares en coordenadas esféricas. Explicítamente

\begin{equation}
	T(x,y,z)
	=
	\big(
		\underbrace{\sqrt{x^2+y^2+z^2}}_{r},
		\underbrace{\arccos(z/r)}_{\theta}
		\underbrace{\arctan(y/x)}_{\varphi},
	\big)
\end{equation}
Usando la regla de la cadena tenemos

\begin{align*}
	D(f\circ T)
	&=
	Df(T)\cdot DT
	\\&=
	\begin{pmatrix}
	\partial_{r} & \partial_{\theta} & \partial_{\varphi}\\
	\end{pmatrix}
	\cdot
	\begin{pmatrix}
	\sin\theta\cos\varphi &
	\sin\theta\sin\varphi &
	\cos\theta
	\\
	\frac{1}{r} \cos\theta\cos\varphi &
	\frac{1}{r} \cos\theta\sin\varphi &
	-\frac{1}{r} \sin\theta
	\\
	-\frac{1}{r} \frac{\sin\varphi}{\sin\theta} &
	\frac{1}{r} \frac{\cos\varphi}{\sin\theta} &
	0
	\end{pmatrix}
\end{align*}

Se concluye que

\begin{align}\label{ap:esfericas:nabla}
	\nabla f
	&=
	\begin{bmatrix}
		\partial_{x} f \\[2pt]
		\partial_{y} f \\[2pt]
		\partial_{z} f \\[2pt]
	\end{bmatrix}
	=
	\begin{bmatrix}
	\sin\theta\cos\varphi \partial_{r}
	+\frac{1}{r} \cos\theta\cos\varphi \partial_{\theta}
	-\frac{1}{r} \frac{\sin\varphi}{\sin\theta} \partial_{\varphi}
	\\[2pt]
	\sin\theta\sin\varphi \partial_{r}
	+\frac{1}{r} \cos\theta\sin\varphi \partial_{\theta}
	+\frac{1}{r} \frac{\cos\varphi}{\sin\theta} \partial_{\varphi}
	\\[2pt]
	\cos\theta \partial_{r}
	-\frac{1}{r} \sin\theta \partial_{\theta}
	\end{bmatrix}
\end{align}

\subsection{Laplaciano escalar}%
\label{ap:esfericas:laplaciano}

En esta sección derivaremos el operador \(\nabla^2\) en coordenadas esféricas. Recordar que para
\(f\colon \R^3 \to \C\) se define su Laplaciano como

\begin{equation}
	\nabla^2 f = \partial^2_{x} f + \partial^2_{y} f + \partial^2_{z} f
.\end{equation}
Veamos cada sumando por separado y después juntaremos todo. Usaremos las
fórmulas derivadas en la sección anterior. Por simplicidad, omitiremos la
función e interpretaremos \(\partial_{\cdot}\) como un operador.

Comenzamos con la derivada con respecto a \(x\)
\begin{align*}
	\partial^2_{x}
	&=
	\partial_{x}
	\underbrace{
	\left(
		\sin\theta\cos\varphi \partial_{r}
		+
		\frac{1}{r} \cos\theta\cos\varphi \partial_{\theta}
		-
		\frac{1}{r} \frac{\sin\varphi}{\sin\theta} \partial_{\varphi}
	\right)
	}_{\eqqcolon \zeta}
	\\&=
	\sin\theta\cos\varphi \partial_{r}(\zeta)
	+
	\frac{1}{r} \cos\theta\cos\varphi \partial_{\theta} (\zeta)
	-
	\frac{1}{r} \frac{\sin\varphi}{\sin\theta} \partial_{\varphi} (\zeta)
\end{align*}

Aplicando los operadores a \(\zeta\) nos queda

\begin{align*}
	\partial_{r}(\zeta)
	&=
	\sin\theta \cos\varphi
	\partial^2_r
	+
	\cos\theta \cos\varphi
	\left[
		-\frac{1}{r^2} \partial_{\theta}
		+
		\frac{1}{r} \partial_{r} \partial_{\theta}
	\right]
	-
	\frac{\sin\varphi}{\sin\theta}
	\left[
		-\frac{1}{r^2} \partial_{\varphi}
		+
		\frac{1}{r} \partial_{r} \partial_{\varphi}
	\right]
	\\
	\partial_{\theta}(\zeta)
	&=
	\cos\varphi
	\left[
		\cos\theta \partial_{r}
		+
		\sin\theta \partial_{\theta} \partial_{r}
	\right]
	+
	\frac{1}{r} \cos\varphi
	\left[
		-\sin\theta \partial_{\theta}
		+
		\cos\theta \partial^2_{\theta}
	\right]
	- \frac{1}{r} \sin\varphi
	\left[
		-\frac{\cos\theta}{\sin^2\theta} \partial_{\varphi}
		+
		\frac{1}{\sin\theta} \partial_{\theta} \partial_{\varphi}
	\right]
	\\
	\partial_{\varphi} (\zeta)
	&=
	\sin\theta
	\left[
		-\sin\varphi \partial_{r}
		+
		\cos\varphi \partial_{\varphi} \partial_{r}
	\right]
	+
	\frac{1}{r} \cos\theta
	\left[
		-\sin\varphi \partial_{\theta}
		+
		\cos\varphi \partial_{\varphi} \partial_{\theta}
	\right]
	- \frac{1}{r\sin\theta}
	\left[
		\cos\varphi \partial_{\varphi}
		+
		\sin\varphi \partial^2_{\varphi}
	\right]
\end{align*}

Juntando todo y agrupando coeficientes se tiene

\begin{align*}
	\partial^2_x
	&=
	\left[
		\sin^2\theta \cos^2\varphi
	\right]
	\partial^2_{r}
	+
	\left[
		\frac{1}{r^2} \cos^2\theta \cos^2\varphi
	\right]
	\partial^2_{\theta}
	+
	\left[
		\frac{1}{r^2} \frac{\sin^2\varphi}{\sin^2\theta}
	\right]
	\partial^2_{\varphi}
	\\&+
	\left[
		\frac{1}{r} \cos^2\theta \cos^2\varphi
		+
		\frac{1}{r} \sin^2\varphi
	\right]
	\partial_{r}
	+
	\left[
		-\frac{2}{r^2} \sin\theta \cos\theta \cos^2\varphi
		+
		\frac{1}{r^2} \frac{\cos\theta}{\sin\theta} \sin^2\varphi
	\right]
	\partial_{\theta}
	\\&+
	\left[
		\frac{1}{r^2} \sin\varphi \cos\varphi
		+
		\frac{1}{r^2} \frac{\cos^2\theta}{\sin^2\theta} \sin\varphi \cos\varphi
		+
		\frac{1}{r^2} \frac{\sin\varphi \cos\varphi}{\sin^2 \theta}
	\right]
	\partial_{\varphi}
	+
	\left[
		\frac{2}{r} \sin\theta \cos\theta \cos^2 \varphi
	\right]
	\partial_{r} \partial_{\theta}
	\\&+
	\left[
		-\frac{2}{r} \sin\varphi \cos\varphi
	\right]
	\partial_{r} \partial_{\varphi}
	+
	\left[
		-\frac{2}{r^2} \frac{\cos\theta}{\sin\theta} \sin\varphi \cos\varphi
	\right]
	\partial_{\theta} \partial_{\varphi}
\end{align*}

De manera similar, la derivada con respecto a \(y\)
\begin{align*}
	\partial^2_{y}
	&=
	\partial_{y}
	\underbrace{%
	\left(
		\sin\theta\sin\varphi \partial_{r}
		+ \frac{1}{r} \cos\theta\sin\varphi \partial_{\theta}
		+ \frac{1}{r} \frac{\cos\varphi}{\sin\theta} \partial_{\varphi}
	\right)
	}_{\eqqcolon \eta}
	\\&=
	\sin\theta\sin\varphi \partial_{r} (\eta)
	+\frac{1}{r} \cos\theta\sin\varphi \partial_{\theta} (\eta)
	+\frac{1}{r} \frac{\cos\varphi}{\sin\theta} \partial_{\varphi} (\eta)
\end{align*}

Aplicamos los operadores a \(\eta\).

\begin{align*}
	\partial_{r}(\eta)
	&=
	\sin\theta \sin\varphi \partial^2_r
	+
	\cos\theta \sin\varphi
	\left[
		-\frac{1}{r^2} \partial_{\theta}
		+
		\frac{1}{r} \partial_{r} \partial_{\theta}
	\right]
	+
	\frac{\cos\varphi}{\sin\theta}
	\left[
		-\frac{1}{r^2} \partial_{\varphi}
		+
		\frac{1}{r} \partial_r \partial_{\varphi}
	\right]
	\\
	\partial_{\theta}(\eta)
	&=
	\sin\varphi
	\left[
		\cos\theta \partial_{r}
		+
		\sin\theta \partial_{\theta} \partial_{r}
	\right]
	+
	\frac{1}{r} \sin\varphi
	\left[
		-\sin\theta \partial_{\theta}
		+
		\cos\theta \partial^2_{\theta}
	\right]
	+
	\frac{1}{r} \cos\varphi
	\left[
		-\frac{\cos\theta}{\sin^2\theta} \partial_{\varphi}
		+
		\frac{1}{\sin\theta} \partial_{\theta} \partial_{\varphi}
	\right]
	\\
	\partial_{\varphi} (\eta)
	&=
	\sin\theta
	\left[
		\cos\varphi \partial_{r}
		+
		\sin\varphi \partial_{\varphi} \partial_{r}
	\right]
	+
	\frac{1}{r} \cos\theta
	\left[
		\cos\varphi \partial_{\theta}
		+
		\sin\varphi \partial_{\varphi} \partial_{\theta}
	\right]
	+
	\frac{1}{r} \frac{1}{\sin\theta}
	\left[
		-\sin\varphi \partial_{\varphi}
		+
		\cos\varphi \partial^2_{\varphi}
	\right]
\end{align*}

Todo junto queda

\begin{align*}
	\partial^2_{y}
	&=
	\left[
		\sin^2\theta \sin^2\varphi
	\right]
	\partial^2_r
	+
	\left[
		+\frac{1}{r^2} \cos^2\theta \sin^2\varphi
	\right]
	\partial^2_{\theta}
	+
	\left[
		\frac{1}{r^2} \frac{\cos^2\varphi}{\sin^2\theta}
	\right]
	\partial^2_{\varphi}
	\\&+
	\left[
		\frac{1}{r} \cos^2\theta \sin^2 \varphi
		+
		\frac{1}{r} \cos^2 \varphi
	\right]
	\partial_{r}
	+
	\left[
		-\frac{2}{r^2} \sin\theta \cos\theta \sin^2\varphi
		+
		\frac{1}{r^2} \frac{\cos\theta}{\sin\theta} \cos^2\varphi
	\right]
	\partial_{\theta}
	\\&+
	\left[
		-\frac{1}{r^2} \sin\varphi \cos\varphi
		-\frac{1}{r^2} \frac{\cos^2\theta}{\sin^2\theta} \sin\varphi \cos\varphi
		-\frac{1}{r^2} \frac{\sin\varphi \cos\varphi}{\sin^2\theta}
	\right]
	\partial_{\varphi}
	+
	\left[
		\frac{2}{r} \sin\theta\cos\theta \sin^2\varphi
	\right]
	\partial_{r} \partial_{\theta}
	\\&+
	\left[
		\frac{2}{r} \sin\varphi \cos\varphi
	\right]
	\partial_{r} \partial_{\varphi}
	+
	\left[
		\frac{2}{r^2} \frac{\cos\theta}{\sin\theta} \sin\varphi \cos\varphi
	\right]
	\partial_{\theta} \partial_{\varphi}
\end{align*}

Ahora veamos la derivada con respecto a \(z\)
\begin{align*}
	\partial^2_z
	&=
	\partial_z
	\underbrace{
	\left(
		\cos\theta \partial_{r}
		-
		\frac{1}{r} \sin\theta \partial_{\theta}
	\right)
	}_{\eqqcolon \xi}
	\\&=
	\cos\theta \partial_{r}(\xi)
	-
	\frac{1}{r} \sin\theta \partial_{\theta} (\xi)
\end{align*}

Como antes, aplicamos los operadores a \(\xi\)

\begin{align*}
	\partial_{r} (\xi)
	&=
	\cos\theta \partial^2_r
	-
	\sin\theta
	\left[
		-\frac{1}{r^2} \partial_{\theta}
		+
		\frac{1}{r} \partial_{r} \partial_{\theta}
	\right]
	\\
	\partial_{\theta} (\xi)
	&=
	- \sin\theta \partial_{r}
	+ \cos\theta \partial_{\theta} \partial_{r}
	- \frac{1}{r}
	\left[
		\cos\theta \partial_{\theta}
		+
		\sin\theta \partial^2_{\theta}
	\right]
\end{align*}

Juntando todo y agrupando términos

\begin{align*}
	\partial^2_{z}
	&=
	\left[
		\cos^2\theta
	\right]
	\partial^2_r
	+
	\left[
		\frac{1}{r^2} \sin^2\theta
	\right]
	\partial^2_{\theta}
	+
	\left[
		\frac{1}{r} \sin^2\theta
	\right]
	\partial_{r}
	+
	\left[
		\frac{2}{r^2} \sin\theta \cos\theta
	\right]
	\partial_{\theta}
	+
	\left[
		-\frac{2}{r} \sin\theta \cos\theta
	\right]
	\partial_{r} \partial_{\theta}
\end{align*}

Finalmente, juntamos todos los términos en \(\partial^2_x, \partial^2_y\) y
\(\partial^2_z\). Agrupando por operadores nos deja

\begin{equation}
\label{eq:laplaciano_esfericas_v1}
	\nabla^2
	=
	\partial^2_r
	+
	\frac{1}{r^2}
	\partial^2_{\theta}
	+
	\frac{1}{r^2 \sin^2\theta}
	\partial^2_{\varphi}
	+
	\frac{2}{r}
	\partial_{r}
	+
	\frac{1}{r^2} \frac{\cos\theta}{\sin\theta}
	\partial_{\theta}
\end{equation}

Para dejarlo de la forma estándar (sin derivadas de orden superior explícitas),
notamos que por regla del producto en los términos con derivadas superiores,
se tiene
\begin{equation}
	\partial^2_r + \frac{2}{r} \partial_{r}
	=
	\frac{1}{r^2} \left(r^2 \partial_{r}\right)
\end{equation}
y
\begin{equation}
	\frac{1}{r^2} \partial^2_{\theta} + \frac{1}{r^2} \frac{\cos\theta}{\sin\theta} \partial_{\theta}
	=
	\frac{1}{r^2\sin\theta} \partial_{\theta} \left(\sin\theta \partial_{\theta} \right)
.\end{equation}

Por lo que el Laplaciano se puede expresar de la siguiente forma

\begin{equation}
\label{eq:laplaciano_esfericas_v2}
	\nabla^2
	=
	\frac{1}{r^2} \partial_{r} \left(r^2 \partial_{r} \right)
	+
	\frac{1}{r^2\sin\theta} \partial_{\theta} \left(\sin\theta \partial_{\theta} \right)
	+
	\frac{1}{r^2\sin^2\theta} \partial^2_{\varphi}
\end{equation}

% Apendice funciones especiales
\section{Funciones especials}\label{ap:fn especiales}

\subsection{Funciones esféricas de Bessel}%
\label{ap:funciones especiales:bessel}

Son las dos soluciones \(j_n\) y \(y_n\) linealmente independientes de la ecuación
\begin{displaymath}
\label{eq:bessel:esfericas}
	x^2 \dd{x} y + 2x \d{x} y + \left(x^2 - n(n+1)\right) y = 0
.\end{displaymath}
La solución \(j_n\) se dice de primer tipo, mientras que \(y_n\) es la de
segundo tipo (también llamada \textit{función esférica de Neumann}).

En base a las propiedades expuestas en \textit{Digital Library of Mathematical
Functions}\footnote{https://dlmf.nist.gov/10.49}. Tenemos que
\begin{alignat*}{2}
	j_n(z)
	&=
	A_0 \sin(z - \frac{1}{2} n \pi) 
	+
	A_1 \cos(z - \frac{1}{2} n \pi)
	\\
	y_n(z)
	&=
	- A_0 \cos(z - \frac{1}{2} n \pi)
	+ A_1 \sin(z - \frac{1}{2} n \pi)
\end{alignat*}
donde 
\begin{alignat*}{3}
	A_0 &= 
	\sum_{k=0}^{\lfloor n/2 \rfloor} 
		(-1)^k \frac{a_{2k} (n + \frac{1}{2}) }{z^{2k + 1}}  
	\\
	A_1 &=
	\sum_{k=0}^{\lfloor (n-1)/2 \rfloor}
		(-1)^k \frac{a_{2k+1}(n + \frac{1}{2}) }{z^{2k+2}} 
	\\
	a_k (n + \frac{1}{2})
	&=
	\begin{cases}
		\frac{(n+k)!}{2^k k! (n-k)!} &, k=0,\dots, n, \\
		0 &, k=n+1, n+2, \dots
	\end{cases}
\end{alignat*}

Nótese que \(j_n \vert_{\R}\) es una función real. Usando que 
\begin{displaymath}
	\sin(z) = \frac{e^{\i z} - e^{-\i z}}{2 \i}
	\hspace{1cm}
	\cos(z) = \frac{e^{\i z} + e^{-\i z}}{2} 
\end{displaymath}
Obtenemos que para \(\kappa\in \R\), \(j_n\) y \(y_n\) representan ondas
estacionarias. En efecto,
\begin{displaymath}
	j_n(\kappa r) e^{-\i \omega t}
	=
	\frac{(A_1 + \i A_0)}{2} e^{\i (z - \frac{1}{2} n \pi - \omega t)}
	+
	\frac{(A_1 - \i A_0)}{2} e^{-\i (z - \frac{1}{2} n \pi + \omega t)}
\end{displaymath}
y
\begin{displaymath}
	y_n(\kappa r) e^{-\i \omega t}
	=
	\frac{-A_0 - \i A_1}{2} e^{\i (\kappa r - \frac{1}{2} n \pi - \omega t)}
	+
	\frac{-A_0 + \i A_1}{2} e^{-\i (\kappa r - \frac{1}{2} n \pi + \omega t)} 
\end{displaymath}
Con esto podemos ver que \(h^{(1)}_n\) y \(h^{(2)}_n\) representan ondas
viajeras, pues,
\begin{displaymath}
	h^{(1)}_n(\kappa r) e^{-\i \omega t}
	=
	j_n(\kappa r) e^{-\i \omega t}
	+
	\i y_n(\kappa r) e^{-\i \omega t}
	=
	A_1 e^{\i (\kappa r - \frac{1}{2} n \pi - \omega t)}
	+
	\i A_0 e^{-\i (\kappa r - \frac{1}{2} n \pi - \omega t)}
\end{displaymath}
y
\begin{displaymath}
	h^{(2)}_n(\kappa r) e^{-\i \omega t}
	=
	j_n(\kappa r) e^{-\i \omega t}
	-
	\i y_n(\kappa r) e^{-\i \omega t}
	=
	\i A_0 e^{\i (\kappa r - \frac{1}{2} n \pi - \omega t)}
	+
	A_1 e^{-\i (\kappa r - \frac{1}{2} n \pi + \omega t)}
\end{displaymath}
Nótese que \(h^{(1)}_n\) viajan a \(+r\) y \(h^{(2)}_n\) viajan \(-r\).

\subsection{Funciones de Legendre}
\label{ap:funciones especiales:legendre}

Son las soluciones \(P_n^m\) y \(Q_n^m\) a la ecuación
\begin{displaymath}
\label{eq:legendre}
	(1-x^2) \dd{x} y - 2 x \d{x} y + \left(n(n+1) - \frac{m^2}{1-x^2}\right) y
	= 0.
\end{displaymath}
Se dicen de primer y segundo tipo respectivamente. Cuando \(m=0\) resultan los
polinomios de Legendre, que equivalen a aplicar el algoritmo de Gram-Schmidt
(sin normalizar) a la base polinómica canónica (i.e. \(1, x, x^2, \dots\)) con
el producto interno \(L^2\).
\end{document}
